% !TEX root = ../main.tex

%%%%%%%%%%%%%%%%%%%%%%%%%%%%%%%%%%%%%%%%%
% Uni-Freiburg BEAMER
% Section: Introduction
% 
% Authors:
% Mücahit Kaya
%%%%%%%%%%%%%%%%%%%%%%%%%%%%%%%%%%%%%%%%%

%----------------------------------------
%   SECTION TITLE
%----------------------------------------
\section{Introduction}
\label{sec:introduction}
\frame[plain]{\sectionpage}


%----------------------------------------
%   SUBSECTION CONTENT
%----------------------------------------
\subsection{Standard Model of Particle Physics}
\label{subsec:standartmodel}
\frame[plain]{\subsectionpage}

\begin{frame}{Standard Model}
    Four fundamental forces:
    \pause
    \begin{itemize}
        \item<1-> Strong Nuclear Force
        \pause
        \item<2-> Electromagnetic Force
        \pause
        \item<3-> Weak Nuclear Force
        \pause
        \item<4-> Gravitation (not part of the SM)
    \end{itemize}
\end{frame}

\begin{frame}{Standard Model}
    \begin{figure}
        \centering
        \includegraphics[scale=0.1]{img/SM.png}
        \caption{Standard Model of Particle Physics.}
        \source{wikipedia.org}
        \label{fig:standard model}
    \end{figure}
\end{frame}

\begin{frame}{Vector and Scalar Bosons}
\begin{itemize}
    \item<1-> Different than gluons and photon , W\(^\pm\) and Z\( ^0\) bosons are massive.
    \item<2-> If we naively added mass terms for the W\(^\pm\) and Z\( ^0\) bosons directly, these mass terms would break the gauge invariance.
\end{itemize}
\end{frame}

\subsection{Higgs Mechanism}
\label{subsec:higss mechanism}
\frame[plain]{\subsectionpage}

\begin{frame}{Higgs Mechanism}
    \begin{itemize}
        \item<1-> This problem brings us to here; how do  W\(^\pm\) and Z\( ^0\) bosons have mass?
      
    \end{itemize}
        \pause
        \vspace{0.5cm}
        \centering
        \setlength{\fboxrule}{1pt}
        \fcolorbox{freiBlue}{white}{
            \parbox{0.325\textwidth}{
                 Answer is Higgs mechanism.
            }}
       \pause
       \vspace{0.5cm}
    \begin{itemize}   
        \item<2-> Why do we need to measure W boson and determine Higgs boson mass?
        \pause
        \vspace{0.5cm}
        \centering
        \setlength{\fboxrule}{1pt}
        \fcolorbox{freiBlue}{white}{
            \parbox{0,75\textwidth}{
                 Accurate measurements of these masses test the validity of the SM and help 
                 verify theoretical predictions.
            }}
    \end{itemize} 
\end{frame}
  


\subsection{CERN, LHC and Atlas Detector}
\label{subsec:cern-lhc}
\frame[plain]{\subsectionpage}

\begin{frame} 
    \frametitle{CERN, LHC and Atlas Detector}
    \begin{columns}[T]
        \begin{column}{.44\textwidth}
            \begin{itemize}
                \item European Council for Nuclear Research (in French, \textbf{C}onseil \textbf{E}uropéen pour la \textbf{R}echerche \textbf{N}ucléaire) was founded in 1954.
                \item  Located on the France–Switzerland border.
                \item   \textbf{L}arge  \textbf{H}adron  \textbf{C}ollider (LHC) is the largest human-made hadron collider. 
               
            \end{itemize}
        \end{column}
        \hfill
        \begin{column}{.54\textwidth}
            \begin{figure}
                \centering
                \includegraphics[scale=0.15]{img/CERN.jpg}
                \caption{The Globe of Science and Innovation, CERN.}
                \source{CERN}
                \label{fig:CERN}
            \end{figure}
        \end{column}
    \end{columns}
\end{frame}

\begin{frame}{CERN}
\begin{figure}
        \centering
        \includegraphics[scale=0.18]{img/atlas-meme.png}
        \label{fig:atlas-meme}
    \end{figure}
\end{frame}










